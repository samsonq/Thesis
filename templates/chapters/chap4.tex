%% This is an example first chapter.  You should put chapter/appendix that you
%% write into a separate file, and add a line \include{yourfilename} to
%% main.tex, where `yourfilename.tex' is the name of the chapter/appendix file.
%% You can process specific files by typing their names in at the 
%% \files=
%% prompt when you run the file main.tex through LaTeX.
\chapter{Potentials of Reinforcement Learning and GANs in Finance}

\section{Identifying Statistical Arbitrage Opportunities}

In Chapter 3, we explored the implementation of a GAN-enabled deep reinforcement learning framework to price and hedge options. We can adapt this framework to find instances of options mispricing by identifying under-valued and over-valued options in the market and applying statistical arbitrage trading strategies.

\subsection{Capturing Mean Reversion with GANs}
An essential component of statistical arbitrage strategies is based on the mean-reverting properties of financial time series to take advantage of price inefficiencies and divergence. In a perfectly efficient market, it would be difficult to implement consistent statistical arbitrage strategies because all available information is reflected in the prices of financial instruments and there would be no price inefficiencies, as stated by the efficient market hypothesis (EMH). Although there is a degree of market efficiency, real markets tend not to be perfectly efficient due to various biases and irrationality exhibited by traders. This results in various asset mispricings that can be arbitraged by using properties such as mean reversion. Although the term statistical arbitrage implies there is no risk to the strategy, there is actually a significant amount of risk when deploying statistical arbitrage strategies in the real markets as price movements are often unpredictable. In order to develop robust statistical arbitrage strategies, simply backtesting on historical data may not be sufficient to confirm the strategy's performance, as previously discussed in Chapter 1.
\\ \\
The risks present in traditional statistical arbitrage methods can be addressed by using reinforcement learning and generative adversarial networks. As explored previously in Chapter 2, GANs have the ability to learn and replicate the statistical properties that empirical financial data exhibit. In the GAN-generated data, there are also patterns of price divergences in a wide variety of market scenarios. To make statistical arbitrage models and strategies more robust, it is possible to optimize and evaluate those models and strategies on AI-generated data with the same mean-reverting properties as historical data \cite{stat-arb}. Furthermore, as statistical arbitrage strategies are highly subject to market risks such as changing market conditions, a reinforcement learning framework can potentially improve trading strategies. Since model-free deep reinforcement learning, combined with GAN-based market simulation, has the ability to quickly adapt to different market scenarios, it may be a powerful tool for learning the optimal statistical arbitrage policy compared to statistical models.

\subsection{Improving Pairs Trading}

One popular type of statistical arbitrage strategy is pairs trading which relies on the assumption of mean reversion between two highly-correlated assets. Often, the difficult part of implementing a pairs trading strategy is identifying the correlated assets to simultaneously trade. The GAN-based approach in this study primarily focused on generating univariate financial time-series data that have similar properties to real data. However, this method can be extended to generating multi-variate time-series data that behave similarly to real multi-variate time-series. By simulating large amounts of time-series data of various correlated candidates, the strength of the correlations can be evaluated. Using this approach also provides the benefit of evaluating a pairs trading strategy's robustness to different market scenarios instead of a simple backtest on historical data, which may provide biased results.
\\ \\
Furthermore, there are often various financial and economic constraints in deploying trading strategies. As it may be difficult to account for these factors and large amounts of market data using traditional methods, it may be possible to implement reinforcement learning methods to make this framework more appropriate for realistic market conditions. For example, it is possible to use the described generative AI framework to generate data in a synthetic market environment and allow a deep reinforcement learning agent to learn the optimal policy for trade execution timing and position sizing. This framework would allow the flexibility to include any constraints while trying to optimize an objective function. Furthermore, this reinforcement learning-based strategy would become more robust when trained on realistic synthetic market data.

\section{Developing Robust Risk Modeling Frameworks}

\subsection{Stress Testing using GAN-Based Market Simulation}

The ability to simulate the possibilities and consequences of extreme market conditions is crucial for risk management. As the future is always unpredictable, Monte Carlo simulations have been widely adopted for simulating possible market scenarios. However, as previously discussed in Chapters 1 and 2, Monte Carlo simulations have limitations in simulating and representing market data. Instead, GANs are more capable of capturing true market dynamics by learning directly from historical market data. A potential application of GANs in risk management practices is to use GAN-based market simulations in stress testing and market risk assessment. GANs could be used to generate scenarios for extreme market conditions, such as severe stock price declines or sharp increases in interest rates. The consequences of market crashes are often unpredictable and hard to model with simple probabilistic simulations, but extreme market conditions can be emulated by GANs learning from, for example, market data during the 2008 financial crisis, or the COVID-19 pandemic. The market simulations generated by the GAN could then be used to stress test a portfolio and see how it would perform under these market conditions.

\subsection{Value-at-Risk using GAN-Based Approach}

A commonly used risk model to assess portfolio tail risks is Value-at-Risk (VaR) which involves implementing a confidence interval on return data to gauge return risks. The three most common approaches to building VaR models include historical simulation, delta-normal method, and Monte Carlo simulation. However, both delta-normal and Monte Carlo simulations are parametric methods that require the assumption about underlying return distributions. As previously discussed in Chapter 1, these parametric methods have limitations and may sometimes be inaccurate in modeling asset returns. For example, the delta-neutral method assumes log returns are normally distributed and estimates a normal distribution using this assumption, and Monte Carlo simulations require assumptions about underlying price movement dynamics. The historical simulation method makes use of past market data to build a confidence interval. However, also previously discussed in Chapter 1, historical data has limitations since it is scarce and doesn't capture enough market scenarios. 
\\ \\
The benefit of using market simulations is the wide range of market scenarios that are generated. Instead of using traditional Monte Carlo methods, the GAN-based market simulation proposed in Chapter 2 can be used to generate various paths of asset returns. As the GANs are trained on historical market data and try to learn the underlying distribution, GAN-based market simulations carry the benefits of both the historical and Monte Carlo simulation methods. The AI-generated data will not only resemble real market data and have similar properties, but the GAN is able to generate an abundance of data, as shown in Figure 2-7. Augmenting traditional risk management tools like Value-at-Risk can potentially make the models more robust to changing market conditions and yield more precise results.