%% This is an example first chapter.  You should put chapter/appendix that you
%% write into a separate file, and add a line \include{yourfilename} to
%% main.tex, where `yourfilename.tex' is the name of the chapter/appendix file.
%% You can process specific files by typing their names in at the 
%% \files=
%% prompt when you run the file main.tex through LaTeX.
\chapter{Introduction}

Derivatives pricing has been a significant field in modern finance as investment managers seek to understand the factors that impact derivative contracts and the underlying, and make trading and hedging decisions on those insights. 

\section{Background of Derivatives Pricing Methods}

Many derivatives pricing models have been created to measure the intrinsic value of derivatives based on factors such as the underlying asset volatility, expiration time, and so on. A notable framework for options pricing known as the Black-Scholes-Merton (BSM) model, developed by Fischer Black, Myron Scholes, and Robert Merton, won the Nobel Prize.

\section{Research Motivations}\label{ch1:sec}

Despite the elegance and theoretically-sound logic of modern pricing models such as the BSM model, these models depend on many market condition assumptions that are unrealistic in the real world. For example, 

\subsection{Market Frictions and Incomplete Markets}
\subsection{Reinforcement Learning}

% This is an example of how you would use tgrind to include an example
% of source code; it is commented out in this template since the code
% example file does not exist.  To use it, you need to remove the '%' on the
% beginning of the line, and insert your own information in the call.
%
%\tgrind[htbp]{code/be.s.tex}{Block Exponent}{opt:be}

\section{Objective}

Quisque sed ultrices leo. Donec vestibulum auctor nibh, at faucibus libero mollis in. Quisque massa lorem, feugiat a lectus in, lobortis volutpat lectus. Donec accumsan dui erat, eu tempor tortor facilisis sed. Nulla ullamcorper augue et sapien dapibus, quis bibendum velit porta. Nullam mattis vehicula tortor porttitor porta. Interdum et malesuada fames ac ante ipsum primis in faucibus. Praesent suscipit, lorem vel viverra rhoncus, turpis orci dignissim dui, bibendum pulvinar justo sem vel lorem. Nam porttitor mollis tristique. Aliquam rhoncus magna quis nisl varius mattis. Sed rhoncus, diam in gravida iaculis, mauris tellus imperdiet turpis, at porttitor est leo vel velit. Praesent faucibus ornare sodales. Sed eu lorem purus.  

\subsection{Dynamic Optimization Problem}

Nullam rhoncus posuere lacus, id volutpat nisi pulvinar viverra. Quisque quis ultricies ante. Duis sollicitudin sapien nec consequat vehicula. Vestibulum convallis erat in arcu aliquam eleifend. Nunc scelerisque lorem non luctus sodales. Curabitur eleifend odio et sagittis semper. Praesent sodales, diam nec vulputate iaculis, neque leo consectetur nunc, a luctus lacus purus et dui. Sed sit amet tortor ullamcorper, malesuada libero quis, imperdiet tortor. Cras tempor blandit massa, sit amet molestie sapien tincidunt quis. Nullam hendrerit venenatis massa, sed lacinia ligula tincidunt vitae.

Nam efficitur et lacus sed eleifend. Aenean quis ipsum eget leo ultrices ornare. Nullam rhoncus ante odio, at dignissim neque posuere eu. Pellentesque sodales tortor est, nec egestas sapien mollis quis. In lectus sapien, pellentesque congue erat consequat, hendrerit aliquet elit. Pellentesque eleifend purus ac diam bibendum, ac auctor ipsum posuere. Cras suscipit leo nec velit fermentum, id varius erat eleifend. Proin sagittis purus id ante lacinia, et congue eros tincidunt. Pellentesque at cursus tellus. Quisque id semper nunc. Quisque viverra a ex at ullamcorper. Morbi mollis erat at ex viverra fringilla. Proin ante dolor, dignissim sodales nisl ac, finibus egestas urna.

Nulla porta urna at pulvinar consectetur. Pellentesque suscipit, neque vitae ultricies rutrum, eros tellus iaculis dui, nec pulvinar justo nibh eu urna. Ut euismod massa nisi, et bibendum risus placerat quis. Integer pretium nulla id risus lobortis laoreet. Aenean quis quam fringilla, elementum odio non, lacinia purus. Vestibulum dui sapien, mollis sit amet massa vel, egestas faucibus velit. Phasellus non justo ut ante vestibulum dictum. Nam in nibh et libero malesuada aliquet. Donec in ex in magna luctus volutpat.

Sed quis dapibus libero. Curabitur id finibus nulla, sed semper felis. Proin dapibus nulla interdum, bibendum tortor et, blandit sapien. Etiam pretium tristique tortor non lacinia. Aliquam dapibus turpis lorem, sit amet porta ex dignissim vitae. In neque felis, sagittis sed ullamcorper lacinia, lobortis ut turpis. Nam quis aliquet justo. Nam eros mi, aliquam vel massa ac, ornare dignissim erat.  This is done by using some combination of
\begin{eqnarray*}
a_i & = & a_j + a_k \\
a_i & = & 2a_j + a_k \\
a_i & = & 4a_j + a_k \\
a_i & = & 8a_j + a_k \\
a_i & = & a_j - a_k \\
a_i & = & a_j \ll m \mbox{shift}
\end{eqnarray*}
instead of the multiplication.  For example, you could use:
\begin{eqnarray*}
r & = & 4s + s\\
r & = & r + r
\end{eqnarray*}
Or by xx:
\begin{eqnarray*}
t & = & 2s + s \\
r & = & 2t + s \\
r & = & 8r + t
\end{eqnarray*}
Cras pharetra ligula nec lectus bibendum, euismod mattis purus cursus. Nullam ut mi molestie purus ultricies lacinia. Phasellus sed orci ac lacus convallis vestibulum. Quisque id nulla ut ipsum finibus vehicula. Curabitur scelerisque erat lobortis, dapibus purus eget, faucibus sapien. Nam enim leo, faucibus id ante sed, fringilla luctus eros. Morbi vulputate, purus at commodo aliquet, turpis dolor sollicitudin libero, id vehicula risus dui sit amet nulla. Sed auctor efficitur urna. Praesent sagittis tellus ac velit vestibulum dignissim. Vivamus justo enim, pellentesque eu posuere id, mattis vitae felis. Aliquam id tincidunt diam. Class aptent taciti sociosqu ad litora torquent per conubia nostra, per inceptos himenaeos. Pellentesque habitant morbi tristique senectus et netus et malesuada fames ac turpis egestas.