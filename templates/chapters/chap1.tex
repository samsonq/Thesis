%% This is an example first chapter.  You should put chapter/appendix that you
%% write into a separate file, and add a line \include{yourfilename} to
%% main.tex, where `yourfilename.tex' is the name of the chapter/appendix file.
%% You can process specific files by typing their names in at the 
%% \files=
%% prompt when you run the file main.tex through LaTeX.
\chapter{Introduction}

\section{Background of Derivatives Pricing Methods}

Financial derivatives pricing has been a prominent and important field in modern finance as investment managers seek to understand the factors that impact derivative contract prices and devise trading strategies on those insights. Derivatives have also been used as a tool for hedging to reduce risk in an investor's positions over time. Options, one type of financial derivative, are one of the most popular instruments for investors, trading an average daily of 39 million contracts. Previous work done on options pricing rely heavily on assumptions about the market and underlying stock movement to solve for partial differential equations and derive pricing models. For example, many models rely on the assumption that stock prices move according to a geometric Brownian motion stochastic process. The most popular framework for pricing options is the Black-Scholes Merton (BSM) model \cite{black-scholes}, derived based on those market assumptions to measure the value of derivatives based on factors including the underlying asset volatility, strike price, interest rate, and expiration time. Another method to price options is to run Monte Carlo Simulations on stock price movement, assuming stock price follows geometric Brownian motion paths, and derive option value based on the results of the simulations. These assumptions about the market and underlying, however, do not always hold in the empirical world, and stock movement patterns are heavily influenced by many factors apart from randomness.

\section{Research Motivation}

\subsection{Market Assumptions}

Despite the elegance of the modern theoretically-sound options pricing models such as Black-Scholes, these models depend on many market condition assumptions that are unrealistic in the real world. For example, in the real world, markets are incomplete and also have many frictions. These market frictions can have significant impacts on option prices and interfere with many trading and hedging strategies. For example, transaction costs resulting from market bid-ask spreads are an important factor to consider, but not accounted for by the Black-Scholes model. Furthermore, interest rates and stock prices are affected by a large number of factors that these stochastic pricing models do not account for. These factors reside within the empirical market data as seen by actual stock price movements and market orders, influenced by trading transactions between various participants in the market. In reality, the market microstructure is very complex and is affected by a large number of factors. As a result, the Black-Scholes model and other traditional options pricing methods do not yield exact option prices quoted on the market. 
\\
\\
Nevertheless, the model serves as a good framework for analyzing which factors come into play for option price movement and can be a useful baseline method. Attributes of an option including strike price and maturity as well as the underlying spot price, volatility, and interest rates are all important factors contributing to the option's price. The Black-Scholes model is given by the following partial differential equation and the resulting closed-form solution for the price of a call option with strike price $K$, maturity $t$, underlying spot price $S$, underlying volatility $\sigma$, and risk-free rate $r$:
\begin{equation}
\begin{aligned}
    \frac{\partial C}{\partial t}+\frac{1}{2}\sigma^2S^2\frac{\partial^2C}{\partial S^2}+rS\frac{\partial C}{\partial S}-rC = 0\\
    C=N(d)S-N(d-\sigma\sqrt{t})Ke^{-rt}\\
    d=\frac{\ln{\frac{S}{K}}+(r+\frac{\sigma^2}{2})t}{\sigma\sqrt{t}}\\
\end{aligned}
\end{equation}
This study explores how to adapt these conventional options pricing frameworks to develop a model-free deep reinforcement learning-based options pricing and hedging algorithm based on empirical market data without relying on market assumptions.

\subsection{Limitations of Historical Financial Data}
Historical financial data is very useful, as it represents real data observed in the market, but incredibly scarce, especially for applications of machine learning methods. If there was an abundance of historical stock price data that is representative of diverse market conditions, it would be possible to build and train machine learning models that are more robust when deployed live. However, even with daily-frequency market data, assuming there are 252 trading days in a year translates to only 2,520 data points for a 10-year period. This presents a problem with only using historical data to build models. To effectively build models and test for an algorithm's generalizability, it is essential to have a large number of samples to train and evaluate a model on.

\subsubsection{Backtesting Limitations}
Often traders evaluate and test models and trading strategies by backtesting on historical market data. Although backtesting is a popular methodology used for testing trading strategy performances, it has many fundamental flaws that may bias the results and allow traders to believe the particular model or strategy will work once deployed, when in fact it may not.
\\
\\
Backtesting simply uses historical market data to evaluate a strategy's performance but does not account for the effect of the strategy's execution on the response from other market participants. This flaw is fundamental in the backtesting methodology because it uses the same data for evaluation, regardless of the trading strategy being evaluated. However, in real-world markets, the execution of different trading strategies, varying by factors such as order size, may have a significant impact on other market orders placed in response. Consequently, backtesting on historical data may bias the evaluation result and is not representative of how a trading algorithm will actually perform once deployed. Furthermore, the factors that affect the markets are constantly evolving and changing over time. Shifts in economic trends may make backtesting on historical data unfavorable because of the inflexibility of testing conditions. Alternatively, a framework that learns from historical market data and consistently generates new market data with similar properties is more favorable.

\subsubsection{Generating Market Data}
A traditional method for generating market data comes through the use of Monte Carlo simulations, in which a large number of data samples are generated through probabilistic simulations. These simulations are parameter-based and rely on the careful selection of parameters to produce accurate and representative data. However, not only are markets always dynamic and have constantly changing conditions, it is difficult to determine which parameters are most important to include in the simulation. Many derivatives pricing models rely on these assumptions about the distribution of the underlying asset price movement. For example, a popular framework for using Monte Carlo simulations to generate stock price paths assumes stock prices follow geometric Brownian motion processes and price levels are log-normally distributed. The Black-Scholes model assumes the underlying stock follows a geometric Brownian motion process (with expected return $\mu$ and volatility $\sigma$), defined by the following stochastic process:
\begin{equation}
\begin{aligned}
    \frac{dS}{S} &= \mu dt+\sigma dW\\
    S_T &= S_0*e^{(r-\frac{1}{2}\sigma^2)t+\sigma\sqrt{t}*N(0,1)}
\end{aligned}
\end{equation}
Using this process, Monte Carlo simulations generate a large number of stock price paths based on the specified parameters return volatility $\sigma$ and some randomness. The advantage of this approach is the unbounded ability to generate large amounts of data representative of the vast amount of different possible paths that the stock can take. The disadvantage, however, comes with the parametric assumptions about underlying stock price movement which may not be completely market-realistic. In reality, stock price movements depend on a large number of factors and contain complex properties. Monte Carlo simulations are generally useful for simulating data with a known and constant probability distribution. However, in the case of financial data, market dynamics and conditions are constantly changing, and it may not be feasible to consistently know the true underlying probability distribution for stock returns.
\\
\\
To overcome some of the limitations of traditional methods like backtesting and Monte Carlo simulations, a possible solution is to create a generative market simulation system that can produce synthetic market data based on historical market data. Adversarial machine learning has the ability to generate fake data that closely resembles real data. In this study, we explore the use of generative adversarial networks (GANs) to learn the market microstructure and systematically generate realistic synthetic market data based on historical market data. This approach is an unsupervised, non-parametric approach as opposed to the use of Monte Carlo simulations. Ultimately, we adapt this framework to train and evaluate the efficacy of our proposed deep reinforcement learning-based pricing and hedging methods.

\section{Objective}

Previous work done on derivatives pricing rely heavily on assumptions about the market and underlying stock movement to be consistent with theoretical standards for solving partial differential equations and deriving pricing models. For example, many simulations rely on the assumption that stock prices move according to a geometric Brownian motion stochastic process. This assumption, however, does not always hold in the empirical world, and underlying price movement patterns are heavily influenced by many factors apart from randomness. The most common and famous framework to compare Monte Carlo pricing simulation results to is the Black-Scholes Merton (BSM) model. This framework is derived from a partial differential equation that also relies on market assumptions.
\\
\\
This study extends previous work and hopes to implement and apply deep reinforcement learning methods to price derivatives on assets in a diverse set of markets with more realistic conditions. This research aims to extend previous work by implementing deep reinforcement learning methods to price financial derivatives and perform dynamic hedging in a diverse set of markets with more realistic market conditions reflected through empirical data. The objective of the deep reinforcement learning approach is to perform better than traditional models on a variety of financial derivatives for multiple asset classes. Then, better hedging strategies can be implemented to achieve higher expected returns and lower volatilities. Multiple state-of-the-art reinforcement learning models will be compared against traditional baseline models for the financial derivatives considered. We explore the implementation of a GAN-based synthetic market to train and evaluate deep hedging agents.

\section{Related Works}

There have been previous works done using deep reinforcement learning for options pricing and hedging. Hans Buehler et al. (2019) researched and developed "Deep Hedging" \cite{deep-hedging} methods using deep reinforcement learning to devise optimal hedging strategies while accounting for market frictions and other factors that impact derivatives prices. In "Deep Hedging", the authors use a synthetic market based on the Heston Model \cite{heston} to generate data with a parametric approach. Furthermore, Igor Halperin implemented a "QLBS" (Q-Learning Black Scholes) \cite{qlbs} model to price and hedge options in a Black-Scholes world based on Q-learning methods. James Hutchinson et al. propose a non-parametric options pricing and hedging approach using traditional machine learning and learning network algorithms \cite{learning-network}. Gordon Ritter et al. propose a framework for using reinforcement learning to hedge derivatives that can be accurately priced \cite{dynamic-replication}. This study extends previous works of deep hedging by training and evaluating the algorithm in a GAN-based market that generates realistic market data representative of and exhibiting similar properties to real market dynamics.