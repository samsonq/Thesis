%% This is an example first chapter.  You should put chapter/appendix that you
%% write into a separate file, and add a line \include{yourfilename} to
%% main.tex, where `yourfilename.tex' is the name of the chapter/appendix file.
%% You can process specific files by typing their names in at the 
%% \files=
%% prompt when you run the file main.tex through LaTeX.
\chapter{Conclusion}
\section{Summary of Analysis}
This research explores how combining artificial intelligence and machine learning methods fueled by data can devise and improve trading and hedging strategies from risk-neutral methods. Deep reinforcement learning is a robust method that can learn the optimal strategy to perform various tasks given different circumstances or market scenarios. It is the technology behind AlphaGo, the A.I. that beat the world Go champion, and is also the backbone to many more amazing applications of A.I. Whereas traditional methods rely on statistical models and stochastic processes to price and hedge options, this study explores a method that relies on empirical data and reinforcement learning algorithms to learn the optimal strategy. This data-driven approach allows the algorithm to learn the specific data-generating process instead of relying on many assumptions about the market and computing options greeks.
\\ \\
In order to train and evaluate this data-driven framework, it is necessary to have an abundance of realistic market data available that can expose the model to various market scenarios and evaluate its robustness to these scenarios. A popular method for testing trading strategies is using historical market data to train a trading strategy and evaluate its performance, known as backtesting. However, this method has many downsides that may bias the results of the test. First, backtesting provides limited data limited to the available market data recorded and the historical events that have occurred. This limitation may not fully be able to evaluate the deep reinforcement learning-based strategy's robustness to new market scenarios. Second, backtesting does not account for the market's response to the execution of the trading strategy being tested. Realistically, when an order is made, there may be a market response by other traders, which is not captured in historical data. These backtesting drawbacks can bias trading strategy evaluation results, and it may not be the case that the strategy will perform well in the future.
\\ \\
To address this issue, the first part of this research explores and implements a systematic data-driven approach to generate realistic synthetic market data from historical data. This method utilizes generative adversarial networks (GANs) to perform adversarial learning to generate new data that looks exactly like real data. This consists of two agents — a generator and a discriminator. The generator takes as input real historical market data and generates new fake data. The discriminator tries to distinguish the generator's output from the real data. Eventually, the generator can learn how to trick the discriminator by generating new realistic market data that can be used for training and testing trading strategies.
\\ \\
The second part of the research focuses on integrating the first part of deep hedging and deep reinforcement learning algorithms with the second part of GAN-based market simulations. The synthetically generated market data is used to train the algorithms by exposing them to a wide variety of market scenarios, representative of those that occurred in the past, as well as those that may occur in the future given the right circumstances. This will improve the robustness of the algorithm by allowing it to learn to adapt to different economic circumstances. The results of this GAN-based deep reinforcement learning framework are compared to traditional models like the Black-Scholes model. We conclude by discussing the efficacy of this systematic approach to developing and testing various data-driven trading strategies through machine learning approaches.

\section{Results and Discussion}
The goal of this study is to explore data-driven machine learning methods that can outperform traditional statistical methods in simulating market data and devising hedging strategies. In Chapter 2, the statistical properties and attributes of GAN-generated time-series market data closely matched those of real market data relative to a parametric Monte Carlo simulation. Based on the similarity of GAN-generated data to real market data relative to Monte Carlo methods, it may be appropriate to use GANs for market simulation purposes. Subsequently, in Chapter 3, deep reinforcement learning methods were explored to price and hedge derivatives in the GAN-based market environment. Based on the distribution of PnLs and volatilities of hedged returns in the two types of market environments, it seems that GANs are able to produce more robust deep hedging agents in solving the hedging problem. There is more work to be done in improving and optimizing these methods into a fully deployable system, but the preliminary results are promising and show the efficacy of this framework in a practical market setting.

\section{Future Work}
This study has primarily focused on developing a systematic proof-of-concept framework to use GAN-generated data to train deep reinforcement learning methods. There is more work to be done for optimizing these methods to produce more precise and robust results and ultimately become deployable in a live trading environment. For example, overfitting is an issue that plagues all forms of machine learning, including the methods explored in this research. GANs can potentially overfit if the generator learns to memorize the market data instead of learning the underlying distribution. RL agents can also overfit by memorizing a specific market environment instead of learning a generalizable hedging policy. Overfitting can cause the system to break when market conditions change, which they often do. Optimizing the system to mitigate overfitting includes hyperparameter tuning, regularization, and other methods.
\\ \\
As GANs consist of deep neural networks, they can learn the complexities of high-dimensional data despite overfitting problems. It is possible to use GANs to simulate multivariate time series data, which appears very frequently in the context of finance. For example, GANs can learn the underlying dynamics of the market microstructure by being trained directly on historical order book data. This makes it possible to train a GAN to generate multi-dimensional time-series data of bid-ask prices. Furthermore, GANs can also handle high-frequency data and can perhaps learn the patterns and dynamics of high-frequency price movements and mimic that behavior in the data it generates. These applications have significant potential for market makers to develop and enhance robust automated trading strategies and market-making algorithms.
\\ \\
The deep hedging framework explored in this study is flexible and robust compared to traditional risk-neutral pricing and hedging methods. Whereas classic options pricing models represent a market with underlying returns following a specific stochastic process, the deep hedging framework can be applied to any sort of market model. Therefore, any type of asset and derivative, along with realistic market conditions such as transaction costs, liquidity risks, and other financial constraints, can be modeled using deep hedging to find the optimal pricing and hedging policy. This study primarily focused on the application to equity markets but can be adjusted to develop hedging strategies in currencies, commodities, fixed income, crypto, etc.
\\
\\
Neural networks and artificial intelligence are known for their black box-like features since many of their predictions are not explainable. This concept applies to GANs and deep reinforcement learning methods in general as well. However, interpretability can potentially reveal useful information about the factors driving the decisions and outputs of GANs and reinforcement learning agents. In the context of GANs, using explainability methods such as activation maps on the generator and discriminator networks can reveal the properties of financial data that the GAN looks at when discriminating and generating new data. This can potentially reveal some unique properties that are generally exhibited by stock returns. Furthermore, in the context of reinforcement learning, using explainability methods can reveal the decision-making process of the deep hedging agent and the factors contributing to its hedging policy. This can potentially reveal the most important market factors and conditions contributing to the hedging decisions made by the agent. Both of these insights may be valuable to traders and investors for analyzing the market and improving investment strategies.

\section{Final Thoughts}
This study explores how GANs can be used as an alternative non-parametric approach to simulate and generate market data as opposed to traditional parametric approaches such as Monte Carlo methods that rely on assumptions about underlying distributions. The systematic market simulation framework is used to train deep hedging agents to find the optimal options pricing and hedging policy. Deep reinforcement learning combined with GAN-based market simulations makes dynamic hedging strategies more robust and precise compared to traditional hedging techniques. 
\\
\\
This study describes just one application of using generative AI to implement and improve other artificial intelligence tools used in solving financial problems. However, there are many more potential applications of these tools to be used in solving other problems in finance. Traditional methods have generally been adopted and practiced due to their simplicity. The increasing access to computing power and the growing abilities of artificial intelligence have enabled machine learning methods to become more accurate and robust, significantly outperforming traditional methods. As AI becomes an increasingly popular research topic among financial services firms, I am excited to see the future growth and applications of AI in the financial industry.