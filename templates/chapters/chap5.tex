%% This is an example first chapter.  You should put chapter/appendix that you
%% write into a separate file, and add a line \include{yourfilename} to
%% main.tex, where `yourfilename.tex' is the name of the chapter/appendix file.
%% You can process specific files by typing their names in at the 
%% \files=
%% prompt when you run the file main.tex through LaTeX.
\chapter{Conclusion}
\section{Summary of Analysis}
This research explores how combining artificial intelligence and machine learning methods fueled by data can devise and improve trading and hedging strategies in the market. Deep reinforcement learning is a robust method that can learn the optimal strategy to perform various tasks given different circumstances or market scenarios. It is the technology behind AlphaGo, the A.I. that beat the world Go champion, and is also the backbone to many more amazing applications of A.I. Whereas traditional methods rely on statistical models and stochastic processes to price and hedge options, this study explores a method that relies on empirical data and reinforcement learning algorithms to learn the optimal strategy. This data-driven approach allows the algorithm to learn the specific data-generating process instead of relying on many assumptions about the market and computing options greeks.
\\ \\
In order to train and evaluate this data-driven framework, it is necessary to have an abundance of market data available that can expose the model to various market scenarios and evaluate its robustness to these scenarios. A popular method for testing trading strategies is known as backtesting where a trading strategy is run against historical market data to see its performance. However, this method has many downsides that may bias the results of the test. First, backtesting provides limited data limited to the available market data recorded and the historical events that have occurred. This limitation may not fully be able to evaluate the deep reinforcement learning-based strategy's robustness to new market scenarios. Second, backtesting does not account for the market's response to the execution of the trading strategy being tested. Realistically, when an order is made, there may be a market response by other traders, which is not captured in historical data. These backtesting drawbacks can bias trading strategy evaluation results, and it may not be the case that the strategy will perform well in the future.
\\ \\
To address this issue, the first part of this research explores and implements a systematic data-driven approach to generate realistic synthetic market data from historical data. This method utilizes generative adversarial networks (GANs) to perform adversarial learning to generate new data that looks exactly like real data. This consists of two agents — a generator and a discriminator. The generator takes as input real historical market data and generates new fake data. The discriminator tries to distinguish the generator's output from the real data. Eventually, the generator can learn how to trick the discriminator by generating new realistic stock and options market data that can be used for training and testing trading strategies.
\\ \\
The second part of the research focuses on integrating the first part of deep hedging and deep reinforcement learning algorithms with the second part of GAN-based market simulations. The synthetically generated market data is used to train the algorithms by exposing them to a wide variety of market scenarios, representative of those that occurred in the past, as well as those that may occur in the future given the right circumstances. This will improve the robustness of the algorithm by allowing it to learn to adapt to different economic circumstances. We conclude by discussing the efficacy of this systematic approach to developing and testing various data-driven trading strategies through machine learning approaches. We also compare the results with traditional models like the Black-Scholes model.

\section{Results and Discussion}

\subsubsection{Artificial Intelligence vs. Stochastic Models}

\section{Future Work}
In this thesis, we have primarily focused on developing a systematic proof-of-concept framework to use GAN-generated data to build and train deep reinforcement learning methods. There is still much work to be done for optimizing these methods to produce more accurate and robust results, and ultimately become deployable in a live trading environment.
\\
\\
There is also great potential in using GAN-generated data for risk management practices, particularly for methods that require market simulations. Value-at-risk, and its variations, is one example of a model that could greatly benefit from using GAN-generated data instead of Monte Carlo simulations.
\\
\\
Furthermore, as GANs consist of deep neural networks, they can learn the complexities of high-dimensional data. It is possible to use GANs to simulate multivariate time series data, which appears very frequently in the context of finance. For example, GANs can learn the underlying dynamics of the market microstructure by being trained directly on historical order book data. This translates into training a GAN to generate multi-dimensional time-series data of bid and ask prices. Furthermore, GANs can also handle high-frequency data and can perhaps learn the patterns and dynamics of high-frequency price movements and mimic that behavior in the data it generates. These applications have significant potential for market makers to develop and enhance robust automated trading strategies and market-making algorithms.
\\
\\
Neural networks and artificial intelligence are known for their black box-like features since many of their predictions are not explainable. This concept applies to GANs and deep reinforcement learning methods in general as well. This research focused primarily on integrating the outputs of these models without worrying about interpretability, but interpretability can 

\section{Final Thoughts}
This study explores how GANs can be used as an alternative non-parametric approach to simulate and generate market data as opposed to traditional parametric approaches such as Monte Carlo methods that rely on assumptions about underlying distributions. The systematic market simulation framework is used to train deep hedging agents to find the optimal options pricing and hedging policy. Deep reinforcement learning combined with GAN-based market simulations makes dynamic hedging strategies more robust and precise compared to traditional Greek hedging. 
\\
\\
This study describes just one application of using generative AI to implement and improve other artificial intelligence tools used in solving financial problems. However, there are many more potential applications of these tools to be used in solving other problems in finance. Traditional methods have generally been adopted and practiced due to their simplicity. The increasing access to computing power and the growing abilities of artificial intelligence have enabled machine learning methods to become more accurate and robust, significantly outperforming traditional methods. As AI becomes an increasingly popular research topic among financial services firms, I am excited to see the future growth and applications of AI in the financial industry.