%% This is an example first chapter.  You should put chapter/appendix that you
%% write into a separate file, and add a line \include{yourfilename} to
%% main.tex, where `yourfilename.tex' is the name of the chapter/appendix file.
%% You can process specific files by typing their names in at the 
%% \files=
%% prompt when you run the file main.tex through LaTeX.
\chapter{Conclusion}
\section{Summary of Results}
This research is divided into three main parts and explores how artificial intelligence and machine learning methods fueled by data can devise and improve trading and hedging strategies in the market. The first part of this research explores and implements deep reinforcement learning algorithms to develop trading and hedging strategies on stocks and options. Deep reinforcement learning is a robust method that can learn the optimal strategy to perform various tasks given different circumstances or market scenarios. It is the technology behind AlphaGo, the A.I. that beat the world Go champion, and is also the backbone to many more amazing applications of A.I. Whereas traditional methods rely on statistical models and stochastic processes to price and hedge options, this study explores a method that relies on empirical data and reinforcement learning algorithms to learn the optimal strategy. This data-driven approach allows the algorithm to learn the specific data-generating process instead of relying on many assumptions about the market and computing options greeks.
\\
\\
In order to train and evaluate this data-driven framework described above, it is necessary to have an abundance of market data available that can expose the model to various market scenarios and evaluate its robustness to these scenarios. A popular method for testing trading strategies is known as backtesting where a trading strategy is run against historical market data to see its performance. However, this method has many downsides that may bias the results of the test. First, backtesting provides limited data limited to the available market data recorded and the historical events that have occurred. This limitation may not fully be able to evaluate the deep reinforcement learning-based strategy's robustness to new market scenarios. Second, backtesting does not account for the market's response to the execution of the trading strategy being tested. Realistically, when an order is made, there may be a market response by other traders, which is not captured in historical data. These backtesting drawbacks can bias trading strategy evaluation results, and it may not be the case that the strategy will perform well in the future.
\\
\\
To address this issue, the second part of this research explores and implements a systematic data-driven approach to generate realistic synthetic market data from historical data. This method utilizes generative adversarial networks (GANs) to perform adversarial learning to generate new data that looks exactly like real data. This consists of two agents — a generator and a discriminator. The generator takes as input real historical market data and generates new fake data. The discriminator tries to distinguish the generator's output from the real data. Eventually, the generator can learn how to trick the discriminator by generating new realistic stock and options market data that can be used for our purposes of training and testing trading strategies.
\\
\\
The third part of the research focuses on integrating the first part of deep hedging and deep reinforcement learning algorithms with the second part of GAN-based market simulations. The synthetically generated market data is used to train the algorithms by exposing it to a wide variety of market scenarios, representative of those that occurred in the past, as well as those that may occur in the future given the right circumstances. This will improve the robustness of the algorithm by allowing it to learn to adapt to different economic circumstances.
\\
\\
We conclude by discussing the efficacy of this systematic approach to developing and testing various data-driven trading strategies through machine learning approaches. We also compare the results with traditional models like Black-Scholes and Heston models.

\subsection{Artificial Intelligence vs. Stochastic Models}

\section{Future Work}
In this thesis, we have primarily focused on developing a systematic proof-of-concept framework to use GAN-generated data to build and train deep reinforcement learning methods. There is still much work to be done for optimizing these methods to produce more accurate and robust results, and ultimately become deployable in a live trading environment.
\\
\\
There is also great potential in using GAN-generated data for risk management practices, particularly for methods that require market simulations. Value-at-risk, and its variations, is one example of a model that could greatly benefit from using GAN-generated data instead of Monte Carlo simulations.
\section{Final Thoughts}
This study explores how GANs can be used as an alternative non-parametric approach to simulate and generate market data as opposed to traditional parametric approaches such as Monte Carlo methods that rely on assumptions about underlying distributions. The systematic market simulation framework is used to train deep hedging agents to find the optimal options pricing and hedging policy. This deep reinforcement learning-based approach combined with GAN-based market simulations makes dynamic hedging strategies more robust and precise compared to traditional Greek hedging.